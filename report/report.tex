\newcommand{\CLASSINPUTtoptextmargin}{2.5cm}
\newcommand{\CLASSINPUTbottomtextmargin}{2cm}
\documentclass[a4paper, 12pt, oneside]{article}
\usepackage{url}
\usepackage[T1]{fontenc}
\usepackage[utf8]{inputenc}
\usepackage{lmodern}
\renewcommand*\familydefault{\sfdefault}
\usepackage{xcolor}
\usepackage{amsmath}
\usepackage{amssymb}
\usepackage{algorithm}
\usepackage{algorithmic}


\newcommand{\red}[1]{\begin{color}{red}#1\end{color}}

%\usepackage{geometry}
%\geometry{letterpaper, margin=1.3in}

\usepackage[acronyms,nonumberlist,nopostdot,nomain,nogroupskip]{glossaries}
\usepackage{tablefootnote}
\usepackage{booktabs}
\usepackage{tabularx}
\usepackage{tikz}
\usepackage{pgfplots}
\pgfplotsset{compat=newest}
\pgfplotsset{plot coordinates/math parser=false}
\newlength\fheight
\newlength\fwidth
\usetikzlibrary{plotmarks,patterns,decorations.pathreplacing,backgrounds,calc,arrows,arrows.meta,spy,matrix}
\usepgfplotslibrary{patchplots,groupplots}
\usepackage{tikzscale}
\usepackage{hyperref}

\usepackage{multirow}
\usepackage{tkz-kiviat}

\renewcommand{\figurename}{Figure}
%\renewcommand{\tablename}{Tab.}
\usepackage[font=small]{subcaption}
\usepackage[font=small]{caption}

\usepackage{mathtools}
\usepackage[affil-it]{authblk}
\renewcommand{\abstractname}{Summary}

\newcommand{\MP}[1]{\color{red}{\textbf{MP says: #1 }}\color{black}}

\usepackage{dblfloatfix}    % To enable figures at the bottom of page
\usepackage{colortbl}

\newacronym{3gpp}{3GPP}{3rd Generation Partnership Project}
\newacronym{adc}{ADC}{Analog to Digital Converter}
\newacronym{afbw}{AFBW}{Average Fading Bandwidth}
\newacronym{5g}{5G}{5th generation}
\newacronym{aimd}{AIMD}{Additive Increase Multiplicative Decrease}
\newacronym{am}{AM}{Acknowledged Mode}
\newacronym{amc}{AMC}{Adaptive Modulation and Coding}
\newacronym{aqm}{AQM}{Active Queue Management}
\newacronym{awgn}{AGWN}{Additive White Gaussian Noise}
\newacronym{balia}{BALIA}{Balanced Link Adaptation}
\newacronym{bsr}{BSR}{Buffer Status Report}
\newacronym{bdp}{BDP}{Bandwidth-Delay Product}
\newacronym{bf}{BF}{Beamforming}
\newacronym{isp}{ISP}{Internet Service Provider}
\newacronym{ngnm}{NGNM}{Next Generation Mobile Networks Alliance}
\newacronym{cc}{CC}{Component Carrier}
\newacronym{drb}{DRB}{Data Radio Bearer}
\newacronym{qos}{QoS}{Quality of Service}
\newacronym{ca}{CA}{Carrier Aggregation}
\newacronym{lc}{LC}{Logical Channel}
\newacronym{rnti}{RNTI}{Radio Network Temporary Identifier}
\newacronym{qci}{QCI}{Quality Class Identifier}
\newacronym{cdf}{CDF}{Cumulative Distribution Function}
\newacronym{cn}{CN}{Core Network}
\newacronym{cqi}{CQI}{Channel Quality Information}
\newacronym{cir}{CIR}{Channel Impulse Response}
\newacronym{cp}{CP}{Control Plane}
\newacronym{csirs}{CSI-RS}{Channel State Information - Reference Signal}
\newacronym{dc}{DC}{Dual Connectivity}
\newacronym{dce}{DCE}{Direct Code Execution}
\newacronym{dci}{DCI}{Downlink Control Information}
\newacronym{dl}{DL}{Downlink}
\newacronym{dmr}{DMR}{Deadline Miss Ratio}
\newacronym{dmrs}{DMRS}{DeModulation Reference Signal}
\newacronym{e2e}{E2E}{End-to-End}
\newacronym{ecn}{ECN}{Explicit Congestion Notification}
\newacronym{edf}{EDF}{Earliest Deadline First}
\newacronym{enb}{eNB}{evolved Node Base}
\newacronym{embb}{eMBB}{enhanced Mobile Broadband}
\newacronym{epc}{EPC}{Evolved Packet Core}
\newacronym{es}{ES}{Edge Server}
\newacronym{fdma}{FDMA}{Frequency Division Multiple Access}
\newacronym{fdd}{FDD}{Frequency Division Duplexing}
\newacronym[firstplural=Radio Access Technologies (RATs)]{rat}{RAT}{Radio Access Technology}
\newacronym{fs}{FS}{Fast Switching}
\newacronym{ftp}{FTP}{File Transfer Protocol}
\newacronym{gnb}{gNB}{Next Generation Node Base}
\newacronym{harq}{HARQ}{Hybrid Automatic Repeat reQuest}
\newacronym{hetnet}{HetNet}{Heterogeneous Network}
\newacronym{hh}{HH}{Hard Handover}
\newacronym{hol}{HOL}{Head-of-Line}
\newacronym{ia}{IA}{Initial Access}
\newacronym{imt}{IMT}{International Mobile Telecommunication}
\newacronym{iot}{IoT}{Internet of Things}
\newacronym{lcr}{LCR}{Level Crossing Rate}
\newacronym{lcf}{LCF}{Level Crossing Frequency}
\newacronym{los}{LoS}{Line-of-Sight}
\newacronym{lte}{LTE}{Long Term Evolution}
\newacronym{m2m}{M2M}{Machine to Machine}
\newacronym{mac}{MAC}{Medium Access Control}
\newacronym{mc}{MC}{Multi-Connectivity}
\newacronym{mcs}{MCS}{Modulation and Coding Scheme}
\newacronym{mec}{MEC}{Mobile Edge Cloud}
\newacronym{mi}{MI}{Mutual Information}
\newacronym{mimo}{MIMO}{Multiple Input, Multiple Output}
\newacronym{mmwave}{mmWave}{millimeter wave}
\newacronym{mptcp}{MPTCP}{Multipath TCP}
\newacronym{mr}{MR}{Maximum Rate}
\newacronym{mss}{MSS}{Maximum Segment Size}
\newacronym{mtd}{MTD}{Machine-Type Device}
\newacronym{mtu}{MTU}{Maximum Transmission Unit}
\newacronym{nfv}{NFV}{Network Function Virtualization}
\newacronym{nlos}{NLoS}{Non-Line-of-Sight}
\newacronym{nr}{NR}{New Radio}
\newacronym{o2i}{O2I}{Outdoor-to-Indoor}
\newacronym{ofdm}{OFDM}{Orthogonal Frequency Division Multiplexing}
\newacronym{pdcch}{PDCCH}{Physical Downlonk Control Channel}
\newacronym{pdcp}{PDCP}{Packet Data Convergence Protocol}
\newacronym{pdsch}{PDSCH}{Physical Downlink Shared Channel}
\newacronym{pdu}{PDU}{Packet Data Unit}
\newacronym{pf}{PF}{Proportional Fair}
\newacronym{pgw}{PGW}{Packet Gateway}
\newacronym{phy}{PHY}{Physical}
\newacronym{pbch}{PBCH}{Physical Broadcast Channel}
\newacronym[plural=\gls{mme}s,firstplural=Mobility Management Entities (MMEs)]{mme}{MME}{Mobility Management Entity}
\newacronym{prb}{PRB}{Physical Resource Block}
\newacronym{pss}{PSS}{Primary Synchronization Signal}
\newacronym{pucch}{PUCCH}{Physical Uplink Control Channel}
\newacronym{pusch}{PUSCH}{Physical Uplink Shared Channel}
\newacronym{rach}{RACH}{Random Access Channel}
\newacronym{ran}{RAN}{Radio Access Network}
\newacronym{red}{RED}{Random Early Detection}
\newacronym{rf}{RF}{Radio Frequency}
\newacronym{rlc}{RLC}{Radio Link Control}
\newacronym{rlf}{RLF}{Radio Link Failure}
\newacronym{rrc}{RRC}{Radio Resource Control}
\newacronym{rrm}{RRM}{Radio Resource Management}
\newacronym{rr}{RR}{Round Robin}
\newacronym{rs}{RS}{Remote Server}
\newacronym{rsrp}{RSRP}{Reference Signal Received Power}
\newacronym{rss}{RSS}{Received Signal Strength}
\newacronym{rtt}{RTT}{Round Trip Time}
\newacronym{rw}{RW}{Receive Window}
\newacronym{rx}{RX}{Receiver}
\newacronym{sa}{SA}{standalone}
\newacronym{sack}{SACK}{Selective Acknowledgment}
\newacronym{sap}{SAP}{Service Access Point}
\newacronym{sch}{SCH}{Secondary Cell Handover}
\newacronym{scoot}{SCOOT}{Split Cycle Offset Optimization Technique}
\newacronym{sdma}{SDMA}{Spatial Division Multiple Access}
\newacronym{sinr}{SINR}{Signal-to-Interference-plus-Noise Ratio}
\newacronym{sir}{SIR}{Signal-to-Interference Ratio}
\newacronym{sm}{SM}{Saturation Mode}
\newacronym{snr}{SNR}{Signal-to-Noise Ratio}
\newacronym{son}{SON}{Self-Organizing Network}
\newacronym{ss}{SS}{Synchronization Signal}
\newacronym{srs}{SRS}{Sounding Reference Signal}
\newacronym{sss}{SSS}{Secondary Synchronization Signal}
\newacronym{tb}{TB}{Transport Block}
\newacronym{tcp}{TCP}{Transmission Control Protocol}
\newacronym{tdd}{TDD}{Time Division Duplexing}
\newacronym{tdma}{TDMA}{Time Division Multiple Access}
\newacronym{tfl}{TfL}{Transport for London}
\newacronym{tm}{TM}{Transparent Mode}
\newacronym{trp}{TRP}{Transmitter Receiver Pair}
\newacronym{tti}{TTI}{Transmission Time Interval}
\newacronym{ttt}{TTT}{Time-to-Trigger}
\newacronym{tx}{TX}{Transmitter}
\newacronym{ue}{UE}{User Equipment}
\newacronym{ul}{UL}{Uplink}
\newacronym{uml}{UML}{Unified Modeling Language}
\newacronym{um}{UM}{Unacknowledged Mode}
\newacronym{uma}{UMa}{Urban Macro}
\newacronym{utc}{UTC}{Urban Traffic Control}
\newacronym{vm}{VM}{Virtual Machine}
\newacronym{rsrq}{RSRQ}{Reference Signal Received Quality}
\newacronym{rssi}{RSSI}{Received Signal Strength Indicator}
\newacronym{crs}{CRS}{Cell Reference Signal}
\newacronym{nsa}{NSA}{Non Stand Alone}
\newacronym{mrdc}{MR-DC}{Multi \gls{rat} \gls{dc}}
\newacronym{endc}{EN-DC}{E-UTRAN-\gls{nr} \gls{dc}}
\newacronym{5gc}{5GC}{5G Core}
\newacronym{si}{SI}{Study Item}
\newacronym{iab}{IAB}{Integrated Access and Backhaul}
\newacronym{wf}{WF}{Wired-first}
\newacronym{hqf}{HQF}{Highest-quality-first}
\newacronym{pa}{PA}{Position-aware}
\newacronym{mlr}{MLR}{Maximum-local-rate}
\newacronym{wbf}{WBF}{Wired Bias Function}
\newacronym{mib}{MIB}{Master Information Block}
\newacronym{sib}{SIB}{Secondary Information Block}
\newacronym{kpi}{KPI}{Key Performance Indicator}
\newacronym{ppp}{PPP}{Poisson Point Process}
\newacronym{mpc}{MPC}{Multi Path Component}
\newacronym{rt}{RT}{Ray Tracer}
\newacronym{aoa}{AoA}{Angle of Arrival}
\newacronym{aod}{AoD}{Angle of Departure}
\newacronym{scm}{SCM}{Spatial Channel Model}
\newacronym{inr}{INR}{Interference to Noise Ratio}
\newacronym{qd}{QD}{Quasi Deterministic}
\newacronym{wlan}{WLAN}{Wireless Local Area Network}
\newacronym{cad}{CAD}{Computer-aided Design}
\newacronym{ap}{AP}{Access Point}
\newacronym{sta}{STA}{Station}
\newacronym{urllc}{URLLC}{Ultra-Reliable Low-Latency Communication}
\newacronym{udp}{UDP}{User Datagram Protocol}
\newacronym{laa}{LAA}{Licensed-Assisted Access}
% tikz styles
\tikzstyle{startstop} = [rectangle, rounded corners, minimum width=2cm, minimum height=0.5cm,text centered, draw=black]
\tikzstyle{io} = [trapezium, trapezium left angle=70, trapezium right angle=110, minimum width=3cm, minimum height=1cm, text centered, draw=black]
\tikzstyle{process} = [rectangle, minimum width=2cm, minimum height=0.5cm, text centered, draw=black, alignb=center]
\tikzstyle{decision} = [ellipse, minimum width=2cm, minimum height=1cm, text centered, draw=black]
\tikzstyle{arrow} = [thick,<->,>=stealth]
\tikzstyle{line} = [thick,>=stealth]
\tikzstyle{darrow} = [thick,<->,>=stealth,dashed]
\tikzstyle{sarrow} = [thick,->,>=stealth]
%\tikzstyle{larrow} = [line width=0.05mm,dashdotted,>=stealth]
\tikzstyle{larrow} = [line width=0.1mm,dashdotted,->,>=stealth]


\makeatletter
\def\grd@save@target#1{%
  \def\grd@target{#1}}
\def\grd@save@start#1{%
  \def\grd@start{#1}}
\tikzset{
  grid with coordinates/.style={
    to path={%
      \pgfextra{%
        \edef\grd@@target{(\tikztotarget)}%
        \tikz@scan@one@point\grd@save@target\grd@@target\relax
        \edef\grd@@start{(\tikztostart)}%
        \tikz@scan@one@point\grd@save@start\grd@@start\relax
        \draw[minor help lines] (\tikztostart) grid (\tikztotarget);
        \draw[major help lines] (\tikztostart) grid (\tikztotarget);
        \grd@start
        \pgfmathsetmacro{\grd@xa}{\the\pgf@x/1cm}
        \pgfmathsetmacro{\grd@ya}{\the\pgf@y/1cm}
        \grd@target
        \pgfmathsetmacro{\grd@xb}{\the\pgf@x/1cm}
        \pgfmathsetmacro{\grd@yb}{\the\pgf@y/1cm}
        \pgfmathsetmacro{\grd@xc}{\grd@xa + \pgfkeysvalueof{/tikz/grid with coordinates/major step x}}
        \pgfmathsetmacro{\grd@yc}{\grd@ya + \pgfkeysvalueof{/tikz/grid with coordinates/major step y}}
        \foreach \x in {\grd@xa,\grd@xc,...,\grd@xb}
        \node[anchor=north] at (\x,\grd@ya) {\pgfmathprintnumber{\x}};
        \foreach \y in {\grd@ya,\grd@yc,...,\grd@yb}
        \node[anchor=east] at (\grd@xa,\y) {\pgfmathprintnumber{\y}};
      }
    }
  },
  minor help lines/.style={
    help lines,
    gray,
    line cap =round,
    xstep=\pgfkeysvalueof{/tikz/grid with coordinates/minor step x},
    ystep=\pgfkeysvalueof{/tikz/grid with coordinates/minor step y}
  },
  major help lines/.style={
    help lines,
    line cap =round,
    line width=\pgfkeysvalueof{/tikz/grid with coordinates/major line width},
    xstep=\pgfkeysvalueof{/tikz/grid with coordinates/major step x},
    ystep=\pgfkeysvalueof{/tikz/grid with coordinates/major step y}
  },
  grid with coordinates/.cd,
  minor step x/.initial=.5,
  minor step y/.initial=.2,
  major step x/.initial=1,
  major step y/.initial=1,
  major line width/.initial=1pt,
}
\makeatother

\def\si{\tikz\fill[scale=0.4](0,.35) -- (.25,0) -- (1,.7) -- (.25,.15) -- cycle;}
\usepackage[a4paper, total={6.2in, 9.2in}]{geometry}
\date{}

\makeglossaries

\begin{document}


\title{\vspace{-2.0cm}\huge A hierarchical SIFT matching \\ implementation in C\texttt{++}} 
\author[1]{\Large Matteo Pagin}
\affil[1]{Department of Information Engineering, University of Padova}

\maketitle

% 3GPP does not want us to use New Radio, so we disable the expansion of the acronym
\glsunset{nr}
\thispagestyle{empty}

\begin{abstract}
	In recent years the usage of binary-valued features such as BRIEF and ORB has experienced a dramatic upsurge, driven by the following advantages over vector-based solutions: faster computation, quicker comparison and a significantly smaller memory footprint. Nevertheless, these binaries features, especially in certain scenarios, lack the descriptive capability of vector-based features such as SIFT, which in this regard are still capable to outperform solutions like BRIEF and ORB. Therefore, the research efforts have been focusing on combining the strengths of these two approaches. In this regard, a fast SIFT matching algorithm is hereby presented and implemented, featuring the binary quantization of SIFT features in conjunction of a hierarchical-based binary matching technique.
  % In this paper, we propose to perform slicing using carrier aggregation, and introduce RoGeR, a cross-carrier scheduling policy
\end{abstract}

\section*{Introduction}
  The task of detecting salient points and computing their description as features has always been of primary importance in computer vision. This problem has been tackled using primarily two approaches, namely binary descriptors such as BRIEF~\cite{calonder2010brief}, ORB~\cite{rublee2011orb} and vector-based solutions like SIFT~\cite{lowe2004distinctive}. While the former category of descriptors is fast to compute and has a very limited memory footprint, the latter exhibit a top of the class robustness wrt scale and rotation changes, on top of unmatched discriminative capabilities over large datasets~\cite{grauman2005efficient, karami2017image}. It follows that transforming vector features into binary ones apparently promises to offer the best of both worlds; this work is in fact based on the same premise. In particular, the presented implementation leverages a median quantization strategy, which manages to obtain satisfying accuracy while exhibiting low computational complexity~\cite{peker2011binary}.\\
  In turn, matching such descriptors has also been another key problem and in such regard the focus of features matching algorithms has also changed over the years: since the size of the available datasets is growing at a staggering pace, the current challenge is to find significant matches in contexts where an exhaustive, linear search is unfeasible. Even though the majority of algorithms sharing such target are based on an approximate nearest neighbour strategy and/or hashing techniques, one of the most promising algorithms for computer vision applications relies on a hierarchical decomposition of the search space~\cite{muja2012fast}. \\
  The purpose of this work is to analyze the performance of the combination of the aforementioned approaches, namely the hierarchical-based matching of median-quantized SIFT features.
  \section*{Median quantization}
  The optimization and integration of these technologies is still an open research challenge, which requires innovations at different layers of the protocol stack. This paper proposes to combine them in a \gls{ran} slicing framework for \glspl{mmwave}, based on carrier aggregation. Notably, we introduce MilliSlice, a cross-carrier scheduling policy that exploits the diversity of the carriers and maximizes their utilization, thus simultaneously guaranteeing high throughput for the \gls{embb} slices and low latency, high reliability for the \gls{urllc} flows.
    The ever increasing number of connected devices and of new, heterogeneous mobile use cases implies that 5G cellular systems will face demanding technical challenges. For example, \gls{urllc} and \gls{embb} scenarios present orthogonal \gls{qos} requirements that the 5G aims to satisfy with a unified \gls{ran} design. Network slicing and \gls{mmwave} communications have been identified as possible enablers for 5G, as they provide, respectively, the necessary scalability and flexibility to adapt the network to each specific use case environment, and low latency and multi-gigabit-per-second wireless links, which tap into a vast, currently unused portion of the spectrum.
    \section*{Fast matching of binary features}
  The optimization and integration of these technologies is still an open research challenge, which requires innovations at different layers of the protocol stack. This paper proposes to combine them in a \gls{ran} slicing framework for \glspl{mmwave}, based on carrier aggregation. Notably, we introduce MilliSlice, a cross-carrier scheduling policy that exploits the diversity of the carriers and maximizes their utilization, thus simultaneously guaranteeing high throughput for the \gls{embb} slices and low latency, high reliability for the \gls{urllc} flows.
    The ever increasing number of connected devices and of new, heterogeneous mobile use cases implies that 5G cellular systems will face demanding technical challenges. For example, \gls{urllc} and \gls{embb} scenarios present orthogonal \gls{qos} requirements that the 5G aims to satisfy with a unified \gls{ran} design. Network slicing and \gls{mmwave} communications have been identified as possible enablers for 5G, as they provide, respectively, the necessary scalability and flexibility to adapt the network to each specific use case environment, and low latency and multi-gigabit-per-second wireless links, which tap into a vast, currently unused portion of the spectrum.
  The optimization and integration of these technologies is still an open research challenge, which requires innovations at different layers of the protocol stack. This paper proposes to combine them in a \gls{ran} slicing framework for \glspl{mmwave}, based on carrier aggregation. Notably, we introduce MilliSlice, a cross-carrier scheduling policy that exploits the diversity of the carriers and maximizes their utilization, thus simultaneously guaranteeing high throughput for the \gls{embb} slices and low latency, high reliability for the \gls{urllc} flows.
    The ever increasing number of connected devices and of new, heterogeneous mobile use cases implies that 5G cellular systems will face demanding technical challenges. For example, \gls{urllc} and \gls{embb} scenarios present orthogonal \gls{qos} requirements that the 5G aims to satisfy with a unified \gls{ran} design. Network slicing and \gls{mmwave} communications have been identified as possible enablers for 5G, as they provide, respectively, the necessary scalability and flexibility to adapt the network to each specific use case environment, and low latency and multi-gigabit-per-second wireless links, which tap into a vast, currently unused portion of the spectrum.
    \section*{Performance evaluation}
  The optimization and integration of these technologies is still an open research challenge, which requires innovations at different layers of the protocol stack. This paper proposes to combine them in a \gls{ran} slicing framework for \glspl{mmwave}, based on carrier aggregation. Notably, we introduce MilliSlice, a cross-carrier scheduling policy that exploits the diversity of the carriers and maximizes their utilization, thus simultaneously guaranteeing high throughput for the \gls{embb} slices and low latency, high reliability for the \gls{urllc} flows.

\bibliographystyle{IEEEtran}
\bibliography{bibl.bib}

\end{document}
